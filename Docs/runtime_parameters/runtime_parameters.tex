\section{ {\tt castro } Namespace}

\label{ch:parameters}


%%%%%%%%%%%%%%%%
% symbol table
%%%%%%%%%%%%%%%%

\begin{landscape}


{\small

\renewcommand{\arraystretch}{1.5}
%
\begin{center}
\begin{longtable}{|l|p{5.25in}|l|}
\caption[castro :  AMR
 parameters]{castro :  AMR
 parameters} \label{table: castro :  AMR
 parameters runtime} \\
%
\hline \multicolumn{1}{|c|}{\textbf{parameter}} & 
       \multicolumn{1}{ c|}{\textbf{description}} & 
       \multicolumn{1}{ c|}{\textbf{default value}} \\ \hline 
\endfirsthead

\multicolumn{3}{c}%
{{\tablename\ \thetable{}---continued}} \\
\hline \multicolumn{1}{|c|}{\textbf{parameter}} & 
       \multicolumn{1}{ c|}{\textbf{description}} & 
       \multicolumn{1}{ c|}{\textbf{default value}} \\ \hline 
\endhead

\multicolumn{3}{|r|}{{\em continued on next page}} \\ \hline
\endfoot

\hline 
\endlastfoot


\rowcolor{tableShade}
\runparamNS{lin\_limit\_state\_interp}{castro} &  how to do limiting of the state data when interpolating 0: only prevent new extrema 1: preserve linear combinations of state variables & 0 \\
\runparamNS{state\_interp\_order}{castro} &  highest order used in interpolation & 1 \\
\rowcolor{tableShade}
\runparamNS{state\_nghost}{castro} &  Number of ghost zones for state data to have. Note that if you are using radiation, choosing this to be zero will be overridden since radiation needs at least one ghost zone. & 0 \\


\end{longtable}
\end{center}

} % ends \small


{\small

\renewcommand{\arraystretch}{1.5}
%
\begin{center}
\begin{longtable}{|l|p{5.25in}|l|}
\caption[castro :  diagnostics
 parameters]{castro :  diagnostics
 parameters} \label{table: castro :  diagnostics
 parameters runtime} \\
%
\hline \multicolumn{1}{|c|}{\textbf{parameter}} & 
       \multicolumn{1}{ c|}{\textbf{description}} & 
       \multicolumn{1}{ c|}{\textbf{default value}} \\ \hline 
\endfirsthead

\multicolumn{3}{c}%
{{\tablename\ \thetable{}---continued}} \\
\hline \multicolumn{1}{|c|}{\textbf{parameter}} & 
       \multicolumn{1}{ c|}{\textbf{description}} & 
       \multicolumn{1}{ c|}{\textbf{default value}} \\ \hline 
\endhead

\multicolumn{3}{|r|}{{\em continued on next page}} \\ \hline
\endfoot

\hline 
\endlastfoot


\rowcolor{tableShade}
\runparamNS{hard\_cfl\_limit}{castro} &  abort if we exceed CFL = 1 over the cource of a timestep & 1 \\
\runparamNS{job\_name}{castro} &  a string describing the simulation that will be copied into the plotfile's {\tt job\_info} file & "" \\
\rowcolor{tableShade}
\runparamNS{print\_fortran\_warnings}{castro} &  display warnings in Fortran90 routines & (0, 1) \\
\runparamNS{sum\_interval}{castro} &  how often (number of coarse timesteps) to compute integral sums (for runtime diagnostics) & -1 \\
\rowcolor{tableShade}
\runparamNS{sum\_per}{castro} &  how often (simulation time) to compute integral sums (for runtime diagnostics) & -1.0e0 \\


\end{longtable}
\end{center}

} % ends \small


{\small

\renewcommand{\arraystretch}{1.5}
%
\begin{center}
\begin{longtable}{|l|p{5.25in}|l|}
\caption[castro :  hydrodynamics
 parameters]{castro :  hydrodynamics
 parameters} \label{table: castro :  hydrodynamics
 parameters runtime} \\
%
\hline \multicolumn{1}{|c|}{\textbf{parameter}} & 
       \multicolumn{1}{ c|}{\textbf{description}} & 
       \multicolumn{1}{ c|}{\textbf{default value}} \\ \hline 
\endfirsthead

\multicolumn{3}{c}%
{{\tablename\ \thetable{}---continued}} \\
\hline \multicolumn{1}{|c|}{\textbf{parameter}} & 
       \multicolumn{1}{ c|}{\textbf{description}} & 
       \multicolumn{1}{ c|}{\textbf{default value}} \\ \hline 
\endhead

\multicolumn{3}{|r|}{{\em continued on next page}} \\ \hline
\endfoot

\hline 
\endlastfoot


\rowcolor{tableShade}
\runparamNS{allow\_negative\_energy}{castro} &  Whether or not to allow internal energy to be less than zero & 0 \\
\runparamNS{allow\_small\_energy}{castro} &  Whether or not to allow the internal energy to be less than the internal energy corresponding to small\_temp & 1 \\
\rowcolor{tableShade}
\runparamNS{cg\_blend}{castro} &  for the Colella \& Glaz Riemann solver, what to do if we do not converge to a solution for the star state. 0 = do nothing; print iterations and exit 1 = revert to the original guess for p-star 2 = do a bisection search for another 2 * cg\_maxiter iterations. & 2 \\
\runparamNS{cg\_maxiter}{castro} &  for the Colella \& Glaz Riemann solver, the maximum number of iterations to take when solving for the star state & 12 \\
\rowcolor{tableShade}
\runparamNS{cg\_tol}{castro} &  for the Colella \& Glaz Riemann solver, the tolerance to demand in finding the star state & 1.0e-5 \\
\runparamNS{density\_reset\_method}{castro} &  Which method to use when resetting a negative/small density 1 = Reset to characteristics of adjacent zone with largest density 2 = Use average of all adjacent zones for all state variables 3 = Reset to the original zone state before the hydro update & 1 \\
\rowcolor{tableShade}
\runparamNS{dual\_energy\_eta1}{castro} &  Threshold value of (E - K) / E such that above eta1, the hydrodynamic pressure is derived from E - K; otherwise, we use the internal energy variable UEINT. & 1.0e0 \\
\runparamNS{dual\_energy\_eta2}{castro} &  Threshold value of (E - K) / E such that above eta2, we update the internal energy variable UEINT to match E - K. Below this, UEINT remains unchanged. & 1.0e-4 \\
\rowcolor{tableShade}
\runparamNS{dual\_energy\_eta3}{castro} &  Threshold value of (E - K) / E such that above eta3, the temperature used in the burning module is derived from E-K; otherwise, we use UEINT. & 1.0e0 \\
\runparamNS{dual\_energy\_update\_E\_from\_e}{castro} &  Allow internal energy resets and temperature flooring to change the total energy variable UEDEN in addition to the internal energy variable UEINT. & 1 \\
\rowcolor{tableShade}
\runparamNS{first\_order\_hydro}{castro} &  set the flattening parameter to zero to force the reconstructed profiles to be flat, resulting in a first-order method & 0 \\
\runparamNS{hse\_interp\_temp}{castro} &  if we are doing HSE boundary conditions, should we get the temperature via interpolation (using model\_parser) or hold it constant? & 0 \\
\rowcolor{tableShade}
\runparamNS{hse\_reflect\_vels}{castro} &  if we are doing HSE boundary conditions, how do we treat the velocity? reflect? or outflow? & 0 \\
\runparamNS{hse\_zero\_vels}{castro} &  if we are doing HSE boundary conditions, do we zero the velocity? & 0 \\
\rowcolor{tableShade}
\runparamNS{hybrid\_riemann}{castro} &  do we drop from our regular Riemann solver to HLL when we are in shocks to avoid the odd-even decoupling instability? & 0 \\
\runparamNS{limit\_fluxes\_on\_small\_dens}{castro} &  Should we limit the density fluxes so that we do not create small densities? & 0 \\
\rowcolor{tableShade}
\runparamNS{ppm\_predict\_gammae}{castro} &  do we construct $\gamma_e = p/(\rho e) + 1$ and bring it to the interfaces for additional thermodynamic information (this is the Colella \& Glaz technique) or do we use $(\rho e)$ (the classic \castro\ behavior).  Note this also uses $\tau = 1/\rho$ instead of $\rho$. & 0 \\
\runparamNS{ppm\_reference\_eigenvectors}{castro} &  do we use the reference state in evaluating the eigenvectors? & 0 \\
\rowcolor{tableShade}
\runparamNS{ppm\_temp\_fix}{castro} &  various methods of giving temperature a larger role in the reconstruction---see Zingale \& Katz 2015 & 0 \\
\runparamNS{riemann\_solver}{castro} &  which Riemann solver do we use: 0: Colella, Glaz, \& Ferguson (a two-shock solver); 1: Colella \& Glaz (a two-shock solver) 2: HLLC & 0 \\
\rowcolor{tableShade}
\runparamNS{small\_dens}{castro} &  the small density cutoff.  Densities below this value will be reset & -1.e200 \\
\runparamNS{small\_temp}{castro} &  the small temperature cutoff.  Temperatures below this value will be reset & -1.e200 \\
\rowcolor{tableShade}
\runparamNS{use\_colglaz}{castro} &  this is deprecated---use {\tt riemann\_solver} instead & -1 \\
\runparamNS{use\_flattening}{castro} &  flatten the reconstructed profiles around shocks to prevent them from becoming too thin & 1 \\
\rowcolor{tableShade}
\runparamNS{xl\_ext\_bc\_type}{castro} &  if we are doing an external -x boundary condition, who do we interpret it? & "" \\
\runparamNS{xr\_ext\_bc\_type}{castro} &  if we are doing an external +x boundary condition, who do we interpret it? & "" \\
\rowcolor{tableShade}
\runparamNS{yl\_ext\_bc\_type}{castro} &  if we are doing an external -y boundary condition, who do we interpret it? & "" \\
\runparamNS{yr\_ext\_bc\_type}{castro} &  if we are doing an external +y boundary condition, who do we interpret it? & "" \\
\rowcolor{tableShade}
\runparamNS{zl\_ext\_bc\_type}{castro} &  if we are doing an external -z boundary condition, who do we interpret it? & "" \\
\runparamNS{zr\_ext\_bc\_type}{castro} &  if we are doing an external +z boundary condition, who do we interpret it? & "" \\


\end{longtable}
\end{center}

} % ends \small


{\small

\renewcommand{\arraystretch}{1.5}
%
\begin{center}
\begin{longtable}{|l|p{5.25in}|l|}
\caption[castro :  parallelization
 parameters]{castro :  parallelization
 parameters} \label{table: castro :  parallelization
 parameters runtime} \\
%
\hline \multicolumn{1}{|c|}{\textbf{parameter}} & 
       \multicolumn{1}{ c|}{\textbf{description}} & 
       \multicolumn{1}{ c|}{\textbf{default value}} \\ \hline 
\endfirsthead

\multicolumn{3}{c}%
{{\tablename\ \thetable{}---continued}} \\
\hline \multicolumn{1}{|c|}{\textbf{parameter}} & 
       \multicolumn{1}{ c|}{\textbf{description}} & 
       \multicolumn{1}{ c|}{\textbf{default value}} \\ \hline 
\endhead

\multicolumn{3}{|r|}{{\em continued on next page}} \\ \hline
\endfoot

\hline 
\endlastfoot


\rowcolor{tableShade}
\runparamNS{do\_acc}{castro} &  determines whether we use accelerators for specific loops & -1 \\


\end{longtable}
\end{center}

} % ends \small


{\small

\renewcommand{\arraystretch}{1.5}
%
\begin{center}
\begin{longtable}{|l|p{5.25in}|l|}
\caption[castro :  timestep control
 parameters]{castro :  timestep control
 parameters} \label{table: castro :  timestep control
 parameters runtime} \\
%
\hline \multicolumn{1}{|c|}{\textbf{parameter}} & 
       \multicolumn{1}{ c|}{\textbf{description}} & 
       \multicolumn{1}{ c|}{\textbf{default value}} \\ \hline 
\endfirsthead

\multicolumn{3}{c}%
{{\tablename\ \thetable{}---continued}} \\
\hline \multicolumn{1}{|c|}{\textbf{parameter}} & 
       \multicolumn{1}{ c|}{\textbf{description}} & 
       \multicolumn{1}{ c|}{\textbf{default value}} \\ \hline 
\endhead

\multicolumn{3}{|r|}{{\em continued on next page}} \\ \hline
\endfoot

\hline 
\endlastfoot


\rowcolor{tableShade}
\runparamNS{cfl}{castro} &  the effective Courant number to use---we will not allow the hydrodynamic waves to cross more than this fraction of a zone over a single timestep & 0.8 \\
\runparamNS{change\_max}{castro} &  the maximum factor by which the timestep can increase from one step to the next. & 1.1 \\
\rowcolor{tableShade}
\runparamNS{dt\_cutoff}{castro} &  the smallest valid timestep---if we go below this, we abort & 0.0 \\
\runparamNS{fixed\_dt}{castro} &  a fixed timestep to use for all steps (negative turns it off) & -1.0 \\
\rowcolor{tableShade}
\runparamNS{init\_shrink}{castro} &  a factor by which to reduce the first timestep from that requested by the timestep estimators & 1.0 \\
\runparamNS{initial\_dt}{castro} &  the initial timestep (negative uses the step returned from the timestep constraints) & -1.0 \\
\rowcolor{tableShade}
\runparamNS{max\_dt}{castro} &  the largest valid timestep---limit all timesteps to be no larger than this & 1.e200 \\


\end{longtable}
\end{center}

} % ends \small


\end{landscape}

%


